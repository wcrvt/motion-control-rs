%以下が文書の先頭に記述するコマンドです.
\documentclass[10pt]{article}
\usepackage[dvipdfmx]{graphicx}
\usepackage{amsmath}
\usepackage{amssymb}
\usepackage[varg]{txfonts}
\usepackage{comment}
\usepackage{multirow}
 \usepackage{bm}
 \usepackage{color}
\special{pdf: minorversion=6}

\setlength{\oddsidemargin}{-10.4mm}    
\setlength{\evensidemargin}{-10.4mm}
\setlength{\textwidth}{180mm}     
\setlength{\topmargin}{-25mm}     
\setlength{\textheight}{265mm} 


\makeatletter
\long\def\@makecaption#1#2{%
\vskip\abovecaptionskip%
\sbox\@tempboxa{#1. #2}% Original
%\sbox\@tempboxa{#2}
\ifdim \wd\@tempboxa >\hsize%
#1. #2\par % Original
%#1 #2\par
\else
\global \@minipagefalse
\hb@xt@\hsize{\hfil\box\@tempboxa\hfil}%
\fi
\vskip\belowcaptionskip}%
\makeatother

\begin{document}
\begin{center}
	\textbf
	{
		\huge{Algorithms for System Identification Tool}\\
	}
 	\vspace{5mm}

\end{center}

\section{Least Squares Method}
Let us consider a system
\begin{align}
y[{\rm k}+1] &= Ax[{\rm k}] + Bu[{\rm k}],
\end{align}
where $A$ and $B$ are the system matrix and input matrix, $u, x, y\in\mathbb{R}$ stand for the input, state, and output, and a prediction system
\begin{align}
\hat{y}[{\rm k}+1] = \theta^{\mathrm T}[{\rm k}]\phi[{\rm k}],
\end{align}
where $\phi$ and $\theta$ mean the parameter vectors, the former is available to sense  and the latter is to be observed.
The purpose of a least squares method is to minimize a cost function
\begin{align}
J(\theta) = \sum \left(y[{\rm k}] - \hat{y}[{\rm k}]\right)^2.	\label{math:lsm_theta}
\end{align}
If a dataset was provided, you can find $\hat{\theta}$ by
\begin{align}
\hat{\theta} =\left(\sum \phi[{\rm k}]\phi^{\mathrm T}[{\rm k}]\right)^{-1}\sum \phi[{\rm k}] y[{\rm k}],
\end{align}

\subsection{ARX model} 
For example, a prediction system using an ARX model is constructed as
\begin{align}
x[{\rm k}+1] = a_0x[{\rm k}] + a_1x[{\rm k}-1] + \cdots + a_{\rm n}x[{\rm k-n}] + b_0u[{\rm k}] + \cdots + b_{\rm m}u[{\rm k-m}],
\end{align}
and the parameters we want are $a_{\rm k}\ ({\rm k}\in[0, {\rm n}])$ and $b_{\rm k}\ ({\rm k}\in[0, {\rm n}])$.
The parameter can be obtained by solving \eqref{math:lsm_theta} setting
\begin{align}
y[k+1] &= x[{\rm k}+1]\\
\theta&=\begin{bmatrix}a_0 & \cdots & a_{\rm n} & b_{0} & \cdots & b_{\rm m}\end{bmatrix}^{\rm T}\\
\phi&=\begin{bmatrix}x[k] & \cdots & x[{\rm k-n}] & u[k] & \cdots & u[{\rm k-m}]\end{bmatrix}^{\rm T}.
\end{align}

\subsection{Whitebox model}
If a whitebox model was provided as
\begin{align}
y[k+1] = a_{\rm 0}x_0[k] + a_{\rm 1} x_1[k] + \cdots + a_{\rm n} x_1[{\rm n}],
\end{align}
where $a_{\rm k}\ ({\rm k}\in[0, {\rm n}])$ are the parameter and $x_{\rm k}\ ({\rm k}\in[0, {\rm n}])$ are available, the parameters can be obtained by solving \eqref{math:lsm_theta} setting
\begin{align}
\theta&=\begin{bmatrix}a_0 & \cdots & a_{\rm n} \end{bmatrix}^{\rm T}\\
\phi&=\begin{bmatrix}x_0[k] & \cdots & x_{\rm n}[k]\end{bmatrix}^{\rm T}.
\end{align}

\subsection{Polynomial model} 
If a polynomial model was provided as
\begin{align}
y[k] = a_{\rm n}x^{\rm n} + a_{\rm n-1}x^{\rm n} + \cdots + a_1x + a_0
\end{align}
where $a_{\rm k}\ ({\rm k}\in[0, {\rm n}])$ are the parameter, the parameters can be obtained by solving \eqref{math:lsm_theta} setting
\begin{align}
\theta&=\begin{bmatrix}a_{\rm n} & \cdots & a_1 & a_0 \end{bmatrix}^{\rm T}\\
\phi&=\begin{bmatrix}x^{\rm n} & \cdots & x & 1 \end{bmatrix}^{\rm T}.
\end{align}

\newpage

\section{Kalman Filter}

Let us consider a state-space representation
\begin{align}
\theta[{\rm k}+1] &= \theta[{\rm k}] + v\\
y[{\rm k}] &= \phi^{\mathrm T}[{\rm k}]\theta[{\rm k}] + w,
\end{align}
where $\phi$ is the observation matrix and $\theta, y, v$, and $w$ denote the parameter to be estimate, available output, process noise, and observation noise.
This system stands for the parameter fluctuate like random walk.
A Kalman filter estimates a parameter following these steps:
\begin{itemize}
\item Prediction step
\begin{align}
\hat{\theta}_{\rm p}[{\rm k}+1] &= \hat{\theta}[{\rm k}]\\
P_{\rm p}[{\rm k} + 1] &= P_{\rm p}[{\rm k}] + Q
\end{align}
\end{itemize}
\begin{itemize}
\item Filtering step
\begin{align}
e[{\rm }k] &= y[{\rm k}] - \phi^{\mathrm T}[{\rm k}]\theta[{\rm k}]\\
S[{\rm k}]&= R + \phi^{\mathrm T}[{\rm k}]P_{\rm p}[k]\phi[{\rm k}]\\
\hat{\theta}[{\rm k}+1] &= \hat{\theta}_{\rm p}[{\rm k}] + P_{\rm p}[{\rm 
k}]\phi[{\rm k}]S[{\rm k}]^{-1}e[{\rm k}]\\
P[{\rm k} + 1] &= P_{\rm p}[{\rm k}] - P_{\rm p}[{\rm 
k}]\phi[{\rm k}]S[{\rm k}]^{-1}\phi^{\mathrm T}[{\rm k}]P_{\rm p}[{\rm 
k}]
\end{align}
\end{itemize}
Here, $P, Q, R$, and $S$ denotes the auto-covariance, covariance of process noise and observation noise, and covariance of observation, and the subscript $_{\rm p}$ stands for the predicted value.



\subsection{ARX model} 
The parameter can be estimated by setting
\begin{align}
y[k+1] &= x[{\rm k}+1]\\
\theta&=\begin{bmatrix}a_0 & \cdots & a_{\rm n} & b_{0} & \cdots & b_{\rm m}\end{bmatrix}^{\rm T}\\
\phi&=\begin{bmatrix}x[k] & \cdots & x[{\rm k-n}] & u[k] & \cdots & u[{\rm k-m}]\end{bmatrix}^{\rm T}.
\end{align}

\subsection{Whitebox model}
The parameter can be estimated by setting
\begin{align}
\theta&=\begin{bmatrix}a_0 & \cdots & a_{\rm n} \end{bmatrix}^{\rm T}\\
\phi&=\begin{bmatrix}x_0[k] & \cdots & x_{\rm n}[k]\end{bmatrix}^{\rm T}.
\end{align}

\subsection{Polynomial model} 
The parameter can be estimated by setting
\begin{align}
\theta&=\begin{bmatrix}a_{\rm n} & \cdots & a_1 & a_0 \end{bmatrix}^{\rm T}\\
\phi&=\begin{bmatrix}x^{\rm n} & \cdots & x & 1 \end{bmatrix}^{\rm T}.
\end{align}



\end{document}